\chapter{Introduction}\label{chapter:introduction}

\graphicspath{{Manuscript/Introduction/Figs/}}

This is a PhD thesis template for pdflatex.
Use \textit{pdflatex} to build latex files.
Don't use TeX+DVI (dvipdfm), or Microtype Package doesn't work.
If you write Japanese characters, use UTF-8 encoding. 

This file specifies the following items:
\begin{itemize}
	\item Font,
	\item Appearance of Text (Hyphenation and Letter-spacing),
	\item Page Layout,
	\item Theorem Environments,
	\item Header and Footer,
	\item Chapter Header,
	\item Table of Contents,
	\item Clickable PDF,
	\item Title Page and PDF-title.
\end{itemize}

The class file was made for Sokendai students.
If you are not a Sokendai student, rewrite the maketitle command in the class file.

We recommend to use \textit{TeX Live} to manage \LaTeX ~packages because this class file uses many packages. 
Before loading the class file, update all packages to the latest version using Tex Live.


%=========================================================================
\section{Managing todo items}
\makeatletter
\newcommand{\verbatimfont}[1]{\def\verbatim@font{#1}}%
\makeatother
\verbatimfont{\rmfamily}

You can make todo notes using the following command:
\begin{verbatim}
\mytodo[inline]{This is a inline todo note} 
\mytodo{This is a todo note at margin}.
\end{verbatim}

In order to display the todo items, use \textit{todo} option.
\begin{verbatim}
\documentclass[todo]{sokendai_thesis}
\end{verbatim}

If you do not want to display the items,
use the following command.
\begin{verbatim}
\documentclass{sokendai_thesis}
\end{verbatim}




